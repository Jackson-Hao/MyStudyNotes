\documentclass[fontset=windows]{article}
\usepackage[margin=1in]{geometry}%设置边距,符合Word设定
\usepackage{ctex}
\usepackage{setspace}
\usepackage{lipsum}
\usepackage{graphicx}%插入图片
\graphicspath{{Figures/}}%文章所用图片在当前目录下的 Figures目录

\usepackage{hyperref} % 对目录生成链接,注:该宏包可能与其他宏包冲突,故放在所有引用的宏包之后
\hypersetup{
	colorlinks=true,
	linkcolor=black,
	citecolor=black,
	bookmarksopen = true, % 展开书签
	bookmarksnumbered = true, % 书签带章节编号
	pdftitle = This is a testfile for vscode, % 标题
	pdfauthor =Jackson Hao} % 作者
\bibliographystyle{plain}% 参考文献引用格式
\newcommand{\upcite}[1]{\textsuperscript{\cite{#1}}}

\renewcommand{\contentsname}{\centerline{Contents}} %经过设置word格式后,将目录标题居中


\title{\heiti\zihao{2} This is a testfile for vscode}
\author{\songti Jackson Hao}
\date{2020.08.02}

\renewcommand{\abstractname}{Abstract}
\renewcommand{\refname}{References}

\begin{document}
	\maketitle
	\thispagestyle{empty}

\begin{abstract}
\noindent This is the abstract. This is the abstract. This is the abstract. This is the abstract. This is the abstract. This is the abstract. This is the abstract. This is the abstract.
\end{abstract}

\newpage

\tableofcontents

\newpage

\section{This is a section}
Hello world! Hello Jackson Hao! As shown in figure \ref{1}.\\
\indent A quick brown fox jumps over the lazy dog. 1234567890 times!
\begin{figure}[htbp]
	\centering
	\includegraphics[scale=0.2]{Ali.jpg}
	\caption{this is Ali}
	\label{1}
\end{figure}



这句话是测试能否进行引用及支持中文\upcite{1}。

\end{document}